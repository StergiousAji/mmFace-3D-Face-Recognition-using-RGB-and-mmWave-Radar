\documentclass{mpaper}

\begin{document}

\title{mmFace: 3D Face Recognition using RGB and Millimetre
Wave Radar}
\author{Stergious Aji}
\matricnum{2546916A}

\maketitle

\begin{abstract}
% According to Simon Peyton Jones, an abstract should address four key questions. First, what is the problem that this paper tackles? Second, why is this an interesting problem? Third, what is the solution this paper proposes? Finally, why is the proposed solution a good one?


\end{abstract}

% This paper outlines the standard template for a final MSci project report submission at the School of Computing Science in the University of Glasgow. In earlier years, MSci students at the School of Computing Science\footnote{\url{https://www.gla.ac.uk/computing}}, University of Glasgow, were expected to produce a full-length dissertation. Now, the requirement is for MSci students to write a paper of up to 14 pages in length, using the supplied \texttt{mpaper} \LaTeX style file.

% The precise structure of an MSci paper is not mandated, but it should probably cover in detail the following aspects of the project.
% \begin{enumerate}
% \item General description of the problem, motivation, relevance
% \item Background information, possibly including a literature survey
% \item Description of approach taken to solve the problem, including high-level design and lower-level implementation details as appropriate
% \item Evaluation, qualitative or quantitative as appropriate
% \item Conclusion, including scope for future work
% \end{enumerate}

\section{Introduction}


\subsection{Motivation}


\subsection{Research Aims}



\section{Background}

\subsection{Related Work}



\section{Methodology}



\section{Evaluation}



\section{Conclusions}



\subsection{Future Work}


% {\bf Acknowledgments.}
% This is optional; it is a location for you to thank people, most probably your family and your supervisor.


\bibliographystyle{unsrt}
\bibliography{l5proj}

\end{document}
